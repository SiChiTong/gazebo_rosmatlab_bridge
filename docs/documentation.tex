\documentclass[letterpaper,10pt]{article}
\usepackage[top = 1in, bottom = 1in, left = 1in, right = 1in]{geometry}
\usepackage[utf8]{inputenc}
\usepackage{fancyvrb}

%opening
\title{Gazebo MATLAB Bridge using Ros serialization \\V1.0} \label{gazebo-matlab-bridge-using-ros-serialization}
\author{Gowtham Garimella}

\begin{document}


\maketitle

\section{Overview}
This package provides a gazebo plugin and a mex interface for fast
communication between MATLAB and Gazebo. The mex interface provides fast
access to link and joint states; easy application of link and joint
efforts; way to set model and joint states. 
\subsection{Prerequisites:}
\begin{itemize}
 \item  ros-hydro 
 \item Gazebo 1.9
 \item ros-hydro-gazebo-packages
\end{itemize}

\section{Package installation}
Clone the package in catkin workspace and source the
setup\_script.bash with the arguments as MATLAB\_ROOT and
ROS\_WORKSPACE.
\begin{Verbatim}[frame=single]
Example Usage: source setup_script.bash /usr/local/MATLAB/R2014a ~/hydro_workspace
\end{Verbatim}
%\textasciitilde{}/hydro\_workspace 
\section{Usage} 
To use the bridge, first run the gazebo with one of the world files provided in ``worlds''
folder. This should load an example model in gazebo with the required
plugin. Then you can run one of the MATLAB scripts provided to control
the models. For a simple example run arm\_closedloop.m and corresponding scene1.world for understanding how plugin works. To load the
world file with gazebo, you can run either of these commands: 
\begin{Verbatim}[frame=single]
 roslaunch gazebo_rosmatlab_bridge basic.launch scene:=scene1 
 ///scene1 can be replaced by any of the scenes from world folder
		  OR
 ///Goto the worlds folder and run:
 gazebo scene1.world #Will load the world file directly
\end{Verbatim}
\textbf{NOTE: Run Gazebo before running MATLAB script always.}

\subsection{Gazebo Plugin Usage}
The world plugin to connect MATLAB with Gazebo is specified as:
\small
\begin{Verbatim}[frame=single]
 <plugin name="gazebo_rosmatlab_bridge" filename="libgazebo_rosmatlab_bridge.so">
    <joints>rear_left_wheel_joint;base_to_steeringblock1;rear_right_wheel_joint</joints>
    <links>carbody</links>
  </plugin>
\end{Verbatim}
\normalsize
The joints of interest for the control problem are specified in a semicolon separated string format. Similarly links are also provided in the same format. If there are multiple links/joints with same name, full scoped names such as (rccar::rear\_left\_wheel\_joint) can be used to distinguish them. When MATLAB client is started, these links and joints show up as ``Available\_Names'' in the GazeboMatlabSimulator class property.

A visual plugin is also provided which allows publishing of desired and current trajectories of different links from MATLAB to Gazebo. This plugin should be loaded in the visual region of a link description in a static model such as ground. See ``models/ground\_plane\_with\_visual\_plugin/model.sdf'' for an example on how to use the visual plugin.

\subsection{MATLAB Documentation}
MATLAB scripts to control gazebo models are provided in matlab\_scripts folder. The documentation for the folder and helper classes can be found in ``docs/matlab\_documentation'' folder.
The description of the documentation files is as follows:

     \begin{table}[h!]
	\begin{tabular}{|p{0.3\textwidth}|p{0.6\textwidth}|}
	  \hline\\
	  Contents\_report.html & Description of the files in the matlab\_scripts folder and what they do\\ \hline
	  GazebomatlabSimulator\_doc.html & Documentation of the various functions and properties for the bridge between MATLAB and Gazebo\\ \hline
	   MatlabLinkInput\_doc.html & Documentation of Link input class \\ \hline
	   MatlabRigidBodyState\_doc.html & Documentation of the Rigid body state class\\ \hline
	\end{tabular}
      \end{table}

These files should be opened from inside MATLAB for the hyperlinks to work properly. For control examples look at the section of examples under ``Closed Loop Controllers'' section. 

\subsection{World Files description}
     \begin{table}[h!]
	\begin{tabular}{|p{0.2\textwidth}|p{0.4\textwidth}|p{0.4\textwidth}|}
	  \hline\\
	  World File & Description & Matlab Script \\ \hline \hline
	  scene1.world & Double Pendulum & arm\_closedloop.m/arm\_computedtorquelaw.m\\ \hline
	  scene2.world & Rcccar model & car\_closedloop.m \\ \hline
	  scene3.world & Wam arm & wam\_servo\_control.m \\ \hline
	  scene4.world & Quadcopter & quad\_bs.m \\ \hline
	  scene5.world & Youbot & \\ \hline
	  scene6.world & Quadcopter with 2DOF arm & \\ \hline
	  scene7.world & AUV vehicle & \\ \hline
	  scene8.world & Satellite & satellite\_test.m\\ \hline
	\end{tabular}
      \end{table}
      
The following section is only for advanced users and is not necessary for students interested in just using the interface.
  
\subsection{Mex Documentaion (Created from the source
file):}\label{mex-documentaion-created-from-the-source-file}

This file provides a mex interface for communicating with Gazebo. This
file should be used along with the world plugin
gazebo\_rosmatlab\_bridge to complete the connection with gazebo. This
interface works by calling string arguments for completing specific
actions. The supported string arguments and example usages are provided
below: 
\begin{itemize}
\item NEW: This creates the bridge. You should always cleanup the
class by calling DELETE after you are done with your simulation. It
provides a pointer to be stored and passed on later.

\begin{Verbatim}[frame=single]
             MATLAB USAGE: A = mex_mmap('new');
\end{Verbatim}

\item
  DELETE: This closes the bridge properly. It should be provided with
  the pointer created using NEW

\begin{Verbatim}[frame=single]
        MATLAB USAGE: mex_mmap('delete',A); %A is the pointer created in above step
\end{Verbatim}
\item
  RESET: This resets the world to its initial state. Useful for
  resetting the simulation after every sample.

\begin{Verbatim}[frame=single]
        MATLAB USAGE: mex_mmap('reset',A); %A is the stored pointer
\end{Verbatim}
\item
  AVAILABLENAMES: This provides the available link and joint names from
  the link. The indices of the links and joints are used below to
  configure the links and joints

\begin{Verbatim}[frame=single]
        MATLAB USAGE: [Link_Names, Joint_Names] = mex_mmap('availablenames',A);
\end{Verbatim}
\item
  CONFIGUREPHYSICS: Configure the physics engine in Gazebo to run
  faster/slower according to your simulation requirements;

\begin{Verbatim}[frame=single]
        MATLAB USAGE: mex_mmap('configurephysics', A, 0.001,2000); 

\end{Verbatim}
        \textit{Explanation}: This runs the physics engine twice the real time speed with a physics timestep of 1 millisecond. The arguments taken are the physics engine timestep 
        Physics engine update frequency: The frequency at which the internal physics engine step is called
\item
  SETMODELSTATE: This sets the complete state of model using Model Pose
  (x,y,z, qw,qx,qy,qz,
  body\_vx,body\_vy,body\_vz,body\_wx,body\_wy,body\_wz){[}13x1{]} in
  world frame and using Joint states(Joint angles, Joint
  Velocities){[}2xn matrix{]}. %\#TODO 
  The Joint Velocities are not supported yet. Gazebo somehow does not set the joint velocities and
  should be replaced with link twists somehow

\begin{Verbatim}[frame=single]
        MATLAB USAGE I: mex_mmap('setmodelstate',h.Mex_data,'Airbotwith2dofarm'...
			,[0 0 0 1 0 0 0 0 0 0 0 0 0],uint32(0:1),[0 pi;0 0]);
        MATLAB USAGE II: mex_mmap('setmodelstate',h.Mex_data,'Airbotwith2dofarm'...
			,[0 0 0 1 0 0 0 0 0 0 0 0 0]);
        MATLAB USAGE III: mex_mmap('setmodelstate',h.Mex_data,'Airbotwith2dofarm'...
			,[],uint32(0:1),[0 pi;0 0]);
\end{Verbatim}
        \textit{Explanation}: The 4th argument is the model pose in world frame. The fifth argument is joint indices (should be uint32) and starting index with 0. The final argument is the actual
        joint angles and joint velocities.
\item
  SETJOINTSTATE: This sets a single joint angle and velocity(%\#TODO
  Velocities not working properly). NOTE: This should not be used for
  moving multiple joints and setmodelstate should be used instead.

\begin{Verbatim}[frame=single]
    MATLAB USAGE: mex_mmap('setjointstate', A, uint32(0) ,[1,0]);
\end{Verbatim}
    \textit{Explanation}: This takes in the zero based joint index as a single scalar which is uint32. The Joint State is just the angle and velocity. 
\item
  RUNSIMULATION: This function runs the gazebo physics engine for
  specified number of steps and provides back the Link and Joint States
  at the steps specified

\begin{Verbatim}[frame=single]
    MATLAB USAGE: [LinkData, JointData] =  mex_mmap('runsimulation', A,  ...
                   JOINTIDS, JOINT_CONTROLS, LINKIDS, LINK_WRENCHES, STEPS);
\end{Verbatim}
    \begin{table}[h!]
    \begin{tabular}{|p{0.2\textwidth}|p{0.7\textwidth}|}
      \hline\\
      A & Stored Pointer \\ \hline
      STEPS& Vector$[N\times 1$ or $1\times N]$  providing number of steps for physics engine. It should start with zero and should be increasing and the last element provides the number of steps the physics engine should take\\ \hline
      JOINTIDS & zero based joint index vector for which controls are provided $[n\times 1$ or $1\times n]$. The indices are obtained from available names described above.\\ \hline
      JOINT\_CONTROLS & Matrix of $[2\times (n\times N)]$ with each $[2xn]$ matrix showing the Joint efforts at that step. The controls are constant in between two steps.\\ \hline %#TODO Add interpolation techniques
      LINKIDS &  zero based link index vector for which link wrenches are provided [nx1 or 1xn]. The indices are obtained from available names described above.\\ \hline
      LINK\_WRENCHES & The Link wrenches are provided as $[6 \times (n \times N)]$. Each $[6\times n]$ matrix provides wrenches for the set of links at that step.\\
      \hline
    \end{tabular}
    \caption{Parameter explanation for runsimulation}
    \end{table}
    If you do not want to apply jointids or joint controls you can replace them by empty matrix. This applies to link ids and wrenches too
    
    \textit{Example System:} Double Pendulum where you want to control the two joints indexed 0, 1. We want to run the simulation for 1 second   
    (Assuming physics timestep is 0.001 this is 1000 steps) with state data every 0.5 seconds. Also we assume that the joint effort is 0.1 for both joints in first half of the trajectory
    and -0.1 in second half of the trajectory. This is called from MATLAB as follows.
    \begin{Verbatim}[frame=single]
    [LinkData,JointData] = mex_mmap('runsimulation',A, uint32([0 1]), ...
			         [0.1 -0.1; 0.1 -0.1], [], [], [0 500 1000]);
    \end{Verbatim}
  NOTE: For more examples look at the examples from matlab\_scripts folder of the package
\end{itemize}

\subsection{KNOWN BUGS:}\label{known-bugs}

There are some known bugs with this interface that need to be fixed. 
\begin{itemize}
 \item You can run the MATLAB scripts only when Gazebo is running. If not it
will get stuck in some infinite loops and MATLAB should be forced to
closed down 
%\item Some times mex interface throws an error (Stream not ready
%to be read). In this case, a temporary solution is to restart gazebo.
%This is caused by the memory maps not syncing properly and to be fixed
\end{itemize}

\end{document}
